\documentclass{article}
\usepackage[spanish]{babel}
\usepackage[utf8]{inputenc}
\usepackage{url}
\usepackage{graphicx}
\usepackage{hyperref}
\setcounter{page}{1} 
\title{Agua y azúcar: la nueva era de la medicina}
\author{Rosalia Piñar Morales}
\begin{document}
\maketitle
\begin{abstract} 

	Repositorio: \url{https://github.com/rosaliapm/proyecto_final.git}
	
	Es agua y el azúcar han sido nuestros aliados desde el comienzo de la vida y ahora han pasado a un segundo puesto como complementos casi, sin aprovechar los beneficios que estos pueden producir. 
	\textit{Palabras clave: agua, azúcar,medicina alternativa.}
	 
\end{abstract}

\section{El agua}

	Desde el inicio de la humanidad, el agua ha sido determinante en nuestra existencia. En todas las culturas a lo largo de la historia es la 'dadora de vida', siempre tiene un papel purificante y salvador. En las tres grandes religiones monoteístas representa la purificación del cuerpo pero también del agua; en la cultura china es la fuente de todo, traída por los dioses; en la mitología nórdica el agua está muy presente en todas sus leyendas.
	Pero hoy en día, siendo esta sociedad más científica y racional, descubrimos que el agua sigue siendo importante en gran medida. Miles de científicos se afanan en buscar agua en otros planetas para demostrar que existe vida más allá de nuestras fronteras planetarias. Entonces, si el agua es capaz de dar la vida en un lugar tan inhóspito como es el espacio, ¿por qué no iba a ser capaz de dar vida aquí en la Tierra e incluso devolverla?
	Pero hay aquí hay un matiz: no toda el agua es buena. Según el doctor en medicina alternativa Masaru Emoto el agua tiene memoria. Por tanto, un agua expuesta a situaciones negativas, sonidos desagradables o con mensajes de odio serían perjudiciales para nuestra salud. incluso las etiquetas usadas en el embotellaje, el ánimo de todas las personas intervinientes en el proceso de comercialización del agua que consumes afecta en gran manera a los beneficios o perjuicios que te produce esta agua. Controlando todo esto y enfocando sólo pensamientos positivos a la hora de beber agua, seríamos capaces de mejorar nuestra salud e incluso de sanarnos por nosotros mismos.
	
\section{El azúcar}
	
	El azúcar ha sido nuestro aliado desde que el hombre habitaba en las cavernas. Usado por nuestros ancestros como sustancia producente de placer, desde entonces ha estado presente en nuestra alimentación diaria. 'Alimento para el cerebro', la glucosa que este posee proporciona la energía necesaria para la concentración a este, indicando su vital función, a pesar de que la sociedad actual le haya impuesto una serie de connotaciones negativas, cuyos efectos no están demostrados totalmente. 
	Por ejemplo, numerosos estudios abalan las propiedades curativas que posee la caña de azúcar: curar resfriados, ablandar abscesos y tumores, tratamiento de afecciones del riñón, estómago y vejiga, aumentar la densidad ósea, problemas en la piel\dots
	
\section{Ampliando beneficios}

	Vistos los beneficios de ambas sustancias, ¿qué pasaría si las combinásemos? La respuesta es simple, sería una combinación de ambos beneficios aportados por separado, al igual que ocurre con los sistemas lineales en matemáticas: el resultado de la suma de dos funciones es la suma de los resultados de cada función por separado. Por tanto, al mezclar agua, correctamente tratada, y azúcar podemos curar muchas de las afecciones que atacan al ser humano que hoy en día se curan mediante medicamentos. 
	Este ahorro de medicamentos supone un gran beneficio para las personas. El abuso de los medicamentos produce efectos adversos en los humanos, al introducir en el organismo elementos químicos altamente tratados, ajenos al cuerpo y que en última instancia este rechazará. Al contrario pasa con el agua y el azúcar: son elementos que te encuentras en la naturaleza de forma normal y que así los consumes, sin pasar por ningún procesado, por lo que el cuerpo los aceptará.
	En esto se basan numerosos movimientos de medicina alternativa que surgen ahora fuertemente, que abogan por una sanidad más natural y menos agresiva.
	
\bibliography{M5_Piñar_Morales_Rosalia}

\end{document}


